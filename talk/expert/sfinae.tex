\subsection[sfinae]{SFINAE}

%https://jguegant.github.io/blogs/tech/sfinae-introduction.html
\begin{frame}[fragile]
  \frametitlecpp[11]{Substitution Failure Is Not An Error (SFINAE)}
  \begin{block}{The main idea}
    \begin{itemize}
    \item substitution replaces template parameters with the provided arguments (types or values)
    \item if it leads to invalid code, do not report an error but try other overloads/specializations
    \end{itemize}
  \end{block}
  \begin{exampleblock}{Example}
    \begin{cppcode*}{}
      template <typename T>
      void f(typename T::type arg) { ... }
      void f(int a) { ... }

      f(1); // Calls void f(int)
    \end{cppcode*}
  \end{exampleblock}
  \begin{alertblock}{}
    Note: SFINAE is largely superseded by concepts in \cpp20
  \end{alertblock}
\end{frame}

\begin{frame}[fragile]
  \frametitlecpp[11]{\texttt{decltype}}
  \begin{block}{The main idea}
    \begin{itemize}
    \item gives the type of the result of an expression
    \item the expression is not evaluated (i.e.\ unevaluated context)
    \item at compile time
    \end{itemize}
  \end{block}
  \begin{exampleblock}{Example}
    \begin{cppcode*}{}
      struct A { double x; };
      A a;
      decltype(a.x) y;        // double
      decltype((a.x)) z = y;  // double& (lvalue)
      decltype(1 + 2u) i = 4; // unsigned int

      template<typename T, typename U>
      auto add(T t, U u) -> decltype(t + u);
      // return type depends on template parameters
    \end{cppcode*}
  \end{exampleblock}
\end{frame}

\begin{frame}[fragile]
  \frametitlecpp[11]{\texttt{declval}}
  \begin{block}{The main idea}
    \begin{itemize}
    \item gives you a reference to a ``fake'' object at compile time
    \item useful for types that cannot easily be constructed
    \item use only in unevaluated contexts, e.g. inside \mintinline{cpp}{decltype}
    \end{itemize}
  \end{block}
  \begin{exampleblock}{}
    \begin{cppcode*}{}
      struct Default {
        int foo() const;
      };
      class NonDefault {
        private: NonDefault();
        public:  int foo() const;
      };
      decltype(Default().foo()) n1 = 1;     // int
      decltype(NonDefault().foo()) n2 = 2; // error
      decltype(std::declval<NonDefault>().foo()) n3 = 3;
    \end{cppcode*}
  \end{exampleblock}
\end{frame}

\begin{frame}[fragile]
  \frametitlecpp[11]{\texttt{true\_type} and \texttt{false\_type}}
  \begin{block}{The main idea}
    \begin{itemize}
    \item encapsulate a compile-time boolean as type
    \item can be inherited
    \end{itemize}
  \end{block}
  \begin{exampleblock}{Example}
    \begin{cppcode*}{}
      struct truth : std::true_type { };

      constexpr bool test = truth::value; // true
      constexpr truth t;
      constexpr bool test = t(); // true
      constexpr bool test = t;   // true
    \end{cppcode*}
  \end{exampleblock}
  \begin{exampleblock}{Possible implementation}
    \begin{cppcode*}{}
      using true_type  = integral_constant<bool, true >;
      using false_type = integral_constant<bool, false>;
    \end{cppcode*}
  \end{exampleblock}
\end{frame}

\begin{frame}[fragile]
  \frametitlecpp[11]{Using SFINAE for introspection}
  \begin{block}{The main idea}
    \begin{itemize}
    \item use a primary template, inheriting from \mintinline{cpp}{false_type}
    \item use a template specialization, inheriting from \mintinline{cpp}{true_type}
    \begin{itemize}
      \item which depends on the feature you want to introspect,
            as part of the context where template argument deduction happens
    \end{itemize}
    \item let SFINAE choose between the two templates
    \end{itemize}
  \end{block}
  \begin{exampleblock}{Example}
    \small
    \begin{cppcode*}{}
      template <typename T, typename = void>
      struct hasFoo : std::false_type {}; // primary template
      template <typename T>
      struct hasFoo<T, decltype(std::declval<T>().foo())>
        : std::true_type {};              // template special.
      struct A{}; struct B{ void foo(); };
      static_assert(!hasFoo<A>::value, "A has no foo()");
      static_assert(hasFoo<B>::value, "B has foo()");
    \end{cppcode*}
  \end{exampleblock}
\end{frame}

\begin{frame}[fragile]
  \frametitlecpp[11]{Not so easy actually...}
  \begin{exampleblockGB}{Example}{https://godbolt.org/z/Y9Ys4hvWq}{\texttt{hasFoo} failing}
    \small
    \begin{cppcode*}{}
      template <typename T, typename = void>
      struct hasFoo : std::false_type {};
      template <typename T>
      struct hasFoo<T, decltype(std::declval<T>().foo())>
        : std::true_type {};

      struct A{};
      struct B{ void foo(); };
      struct C{ int  foo(); };

      static_assert(!hasFoo<A>::value, "A has no foo()");
      static_assert(hasFoo<B>::value, "B has foo()");
      static_assert(!hasFoo<C>::value, "C has no foo()");
      static_assert(hasFoo<C,int>::value, "C has foo()");
    \end{cppcode*}
  \end{exampleblockGB}
\end{frame}

\begin{frame}[fragile]
  \frametitlecpp[17]{Using \texttt{void\_t}}
  \begin{block}{Concept}
    \begin{itemize}
    \item Maps a sequence of given types to \mintinline{cpp}{void}
    \item Introduced in \cpp17, though trivial to implement in \cpp11
    \item Can be used in specializations to check the validity of an expression
    \end{itemize}
  \end{block}
  \begin{block}{Implementation in header type\_traits}
    \begin{cppcode*}{gobble=2}
      template <typename...>
      using void_t = void;
    \end{cppcode*}
  \end{block}
\end{frame}

\begin{frame}[fragile]
  \frametitlecpp[17]{Previous introspection example using \texttt{void\_t}}
    \begin{exampleblockGB}{Example}{https://godbolt.org/z/hfMxndP1x}{\texttt{hasFoo} working}
      \begin{cppcode*}{}
      template <typename T, typename = void>
      struct hasFoo : std::false_type {};

      template <typename T>
      struct hasFoo<T,
         std::void_t<decltype(std::declval<T>().foo())>>
      : std::true_type {};

      struct A{}; struct B{ void foo(); };
      struct C{ int foo(); };

      static_assert(!hasFoo<A>::value,"unexpected foo()");
      static_assert(hasFoo<B>::value, "expected foo()");
      static_assert(hasFoo<C>::value, "expected foo()");
      \end{cppcode*}
    \end{exampleblockGB}
\end{frame}

\begin{frame}[fragile]
  \frametitlecpp[11]{Standard type traits}
  \begin{block}{Standard type traits}
    \begin{itemize}
    \item Don't implement SFINAE introspection for everything,\\
          use the standard library!
    \item Standard type traits in header \mintinline{cpp}{<type_traits>}
    \item They look like \mintinline{cpp}{std::is_*<T>}
    \item Checks at compile time whether \mintinline{cpp}{T} is
    \begin{itemize}
      \item float, signed, final, abstract, default\_constructible, ...
    \end{itemize}
    \item Result in nested boolean constant \mintinline{cpp}{value}
    \item Since \cpp17, variable template versions: \mintinline{cpp}{std::is_*_v<T>}, giving the bool directly
    \end{itemize}
  \end{block}
  \begin{cppcode*}{}
    constexpr bool b = std::is_pointer<int*>::value;
    constexpr bool b = std::is_pointer_v<int*>; // C++17
  \end{cppcode*}
\end{frame}

\begin{frame}[fragile]
  \frametitle{SFINAE and the STL \hfill \cpp11/\cpp14}
  \begin{block}{\texttt{enable\_if} \cpprefLink{https://en.cppreference.com/w/cpp/types/enable_if}}
    \begin{cppcode*}{linenos=false}
      template<bool B, typename T=void>
      struct enable_if;
    \end{cppcode*}
    \begin{itemize}
    \item If \mintinline{cpp}{B} is true, has a nested alias \mintinline{cpp}{type} to type \mintinline{cpp}{T}
    \item otherwise, has no such alias
    \end{itemize}
  \end{block}
  \begin{exampleblock}{Example}
    \begin{cppcode*}{}
      template<bool B, typename T=void>
      struct enable_if {};
      template<typename T>
      struct enable_if<true, T> { using type = T; };

      template<bool B, typename T = void> // C++14
      using enable_if_t = typename enable_if<B, T>::type;
    \end{cppcode*}
  \end{exampleblock}
\end{frame}

\begin{frame}[fragile]
  \begin{exampleblockGB}{Usage example}{https://godbolt.org/z/xse8GM8fq}{\texttt{enable\_if}}
    \small
    \begin{cppcode*}{}
    template <typename T> constexpr bool is_iterable ... ;

    template<typename T,
             typename std::enable_if_t
               <!is_iterable<T>, bool> = true>
    void print(T const& t) {
        std::cout << t << "\n";
    }
    template<typename T,
             typename std::enable_if_t
               <is_iterable<T>, bool> = true>
    void print(T const& t) {
        for (auto const& item : t) {
            std::cout << item << "\n";
        }
    }
    \end{cppcode*}
    Note : using \mintinline{cpp}{if constexpr} would be simpler
  \end{exampleblockGB}
\end{frame}

\begin{frame}[fragile]
  \frametitlecpp[11]{Back to variadic class templates}
  \begin{block}{The \texttt{tuple\_element} trait}
    \begin{cppcode*}{}
      template <size_t I, typename Tuple>
      struct tuple_element;

      template <typename T, typename... Ts>
      struct tuple_element<0, tuple<T, Ts...>> {
        using type = T;
      };

      template <size_t I, typename T, typename... Ts>
      struct tuple_element<I, tuple<T, Ts...>> {
        using type = typename
          tuple_element<I - 1, tuple<Ts...>>::type;
      };
    \end{cppcode*}
  \end{block}
\end{frame}

\begin{frame}[fragile]
  \frametitlecpp[14]{Back to variadic class templates}
  \begin{block}{The tuple get function}
    \begin{cppcode*}{gobble=2}
      template <size_t I, typename... Ts>
      std::enable_if_t<I == 0,
        typename tuple_element<0, tuple<Ts...>>::type&>
      get(tuple<Ts...>& t) {
        return t.m_head;
      }
      template <size_t I, typename T, typename... Ts>
      std::enable_if_t<I != 0,
        typename tuple_element<I - 1, tuple<Ts...>>::type&>
      get(tuple<T, Ts...>& t) {
        return get<I - 1>(static_cast<tuple<Ts...>&>(t));
      }
    \end{cppcode*}
  \end{block}
\end{frame}

\begin{frame}[fragile]
  \frametitlecpp[17]{with if constexpr and return type deduction}
  \begin{block}{The tuple get function}
    \begin{cppcode*}{gobble=2}
      template <size_t I, typename T, typename... Ts>
      auto& get(tuple<T, Ts...>& t) {
        if constexpr(I == 0)
          return t.m_head;
        else
          return get<I - 1>(static_cast<tuple<Ts...>&>(t));
      }
    \end{cppcode*}
  \end{block}
  \begin{goodpractice}{SFINAE vs.\ if constexpr}
    \begin{itemize}
      \item \mintinline{cpp}{if constexpr} can replace SFINAE in many places.
      \item It is usually more readable as well. Use it if you can.
    \end{itemize}
  \end{goodpractice}
\end{frame}
