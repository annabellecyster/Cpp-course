\subsection[Core]{Core syntax and types}

\begin{frame}[fragile]
  \frametitlecpp[98]{Hello World}
  \begin{cppcode}
    #include <iostream>

    // This is a function
    void print(int i) {
      std::cout << "Hello, world " << i << std::endl;
    }

    int main(int argc, char** argv) {
      int n = 3;
      for (int i = 0; i < n; i++) {
        print(i);
      }
      return 0;
    }
  \end{cppcode}
\end{frame}

\begin{frame}[fragile]
  \frametitlecpp[98]{Comments}
  \begin{cppcode}
    // simple comment until end of line
    int i;

    /* multiline comment
     * in case we need to say more
     */
    double /* or something in between */ d;

    /**
     * Best choice : doxygen compatible comments
     * \brief checks whether i is odd
     * \param i input
     * \return true if i is odd, otherwise false
     * \see https://www.doxygen.nl/manual/docblocks.html
     */
    bool isOdd(int i);
  \end{cppcode}
\end{frame}

\begin{frame}[fragile]
  \frametitlecpp[98]{Basic types(1)}
  \begin{cppcode}
    bool b = true;            // boolean, true or false

    char c = 'a';             // min 8 bit integer
                              // may be signed or not
                              // can store an ASCII character
    signed char c = 4;        // min 8 bit signed integer
    unsigned char c = 4;      // min 8 bit unsigned integer

    char* s = "a C string";   // array of chars ended by \0
    string t = "a C++ string";// class provided by the STL

    short int s = -444;       // min 16 bit signed integer
    unsigned short s = 444;   // min 16 bit unsigned integer
    short s = -444;           // int is optional
  \end{cppcode}
\end{frame}
\begin{frame}[fragile]
  \frametitlecpp[98]{Basic types(2)}
  \begin{cppcode}
    int i = -123456;          // min 16, usually 32 bit
    unsigned int i = 1234567; // min 16, usually 32 bit

    long l = 0L               // min 32 bit
    unsigned long l = 0UL;    // min 32 bit

    long long ll = 0LL;          // min 64 bit
    unsigned long long l = 0ULL; // min 64 bit

    float f = 1.23f;          // 32 (1+8+23) bit float
    double d = 1.23E34;       // 64 (1+11+52) bit float
    long double ld = 1.23E34L // min 64 bit float
  \end{cppcode}
\end{frame}

\begin{frame}[fragile]
  \frametitlecpp[98]{Portable numeric types}
  \begin{cppcode}
    #include <cstdint> // defines the following:

    std::int8_t c = -3;     // 8 bit signed integer
    std::uint8_t c = 4;     // 8 bit unsigned integer

    std::int16_t s = -444;  // 16 bit signed integer
    std::uint16_t s = 444;  // 16 bit unsigned integer

    std::int32_t s = -674;  // 32 bit signed integer
    std::uint32_t s = 674;  // 32 bit unsigned integer

    std::int64_t s = -1635; // 64 bit signed integer
    std::uint64_t s = 1635; // 64 bit unsigned int
  \end{cppcode}
\end{frame}

\begin{frame}[fragile]
  \frametitlecpp[98]{Integer literals}
  \begin{cppcode}
    int i = 1234;            // decimal     (base 10)
    int i = 02322;           // octal       (base  8)
    int i = 0x4d2;           // hexadecimal (base 16)
    int i = 0X4D2;           // hexadecimal (base 16)
    int i = 0b10011010010;   // binary      (base  2) C++14

    int i = 123'456'789;     // digit separators, C++14
    int i = 0b100'1101'0010; // digit separators, C++14

    42           // int
    42u,   42U   // unsigned int
    42l,   42L   // long
    42ul,  42UL  // unsigned long
    42ll,  42LL  // long long
    42ull, 42ULL // unsigned long long
  \end{cppcode}
\end{frame}

\begin{frame}[fragile]
  \frametitlecpp[98]{Floating-point literals}
  \begin{cppcode}
    double d = 12.34;
    double d = 12.;
    double d = .34;
    double d = 12e34;           // 12 * 10^34
    double d = 12E34;           // 12 * 10^34
    double d = 12e-34;          // 12 * 10^-34
    double d = 12.34e34;        // 12.34 * 10^34

    double d = 123'456.789'101; // digit separators, C++14

    double d = 0x4d2.4p3;   // hexfloat, 0x4d2.4 * 2^3
                            // = 1234.25 * 2^3 = 9874

    3.14f, 3.14F,  // float
    3.14,  3.14,   // double
    3.14l, 3.14L,  // long double
  \end{cppcode}
\end{frame}

\begin{advanced}
\begin{frame}[fragile]
  \frametitlecpp[98]{Useful aliases}
  \begin{cppcode}
    #include <cstddef> // (and others) defines:

    // unsigned integer, can hold any variable's size
    std::size_t s = sizeof(int);

    #include <cstdint> // defines:

    // signed integer, can hold any diff between two pointers
    std::ptrdiff_t c = &s - &s;

    // signed/unsigned integer, can hold any pointer value
    std::intptr_t i = reinterpret_cast<intptr_t>(&s);
    std::uintptr_t i = reinterpret_cast<uintptr_t>(&s);
    \end{cppcode}
\end{frame}

\begin{frame}[fragile]
    \frametitlecpp[23]{Extended floating-point types}
    \begin{block}{Extended floating-point types}
        \begin{itemize}
            \item \emph{Optional} types, which may be provided
            \item Custom conversion and promotion rules
        \end{itemize}
    \end{block}
    \begin{cppcode*}{gobble=4}
        #include <stdfloat> // may define these:

        std::float16_t  = 3.14f16;  // 16 (1+5+10) bit float
        std::float32_t  = 3.14f32;  // like float
                                    // but different type
        std::float64_t  = 3.14f64;  // like double
                                    // but different type
        std::float128_t = 3.14f128; // 128 (1+15+112) bit float
        std::bfloat_t   = 3.14bf16; // 16 (1+8+7) bit float

        // also F16, F32, F64, F128 or BF16 suffix possible
    \end{cppcode*}
\end{frame}
\end{advanced}
